\section{Reliability and Validity}
Exploring the applicability of deploying recurrent neural networks in predicting small cap stock returns is fundamentally a challenging task, as future development of individual stock returns and the stock market as a whole, is difficult to predict. There are numerous factors that influence the stock market, whether it be fundamental, technical or macroeconomic factors. While the thesis has tried to accommodate several of these complex issues, it is important to discuss the reliability and validity of the resulting findings. Evidence from analyzing portfolio performance suggests that recurrent neural networks can be employed to predict small cap stock returns and achieve substantial excess returns, compared to large cap stocks, the small cap benchmark and the large cap benchmark. However, there exists some uncertainties with the validity and reliability with these findings, where areas including data collection and methodology requires further research to be able to confirm the results. 

\indent\newline
A potential reason why there is such a big difference in predictive performance and portfolio performance is the imbalanced data sets. Having data on nearly twice as many small cap stocks as large cap stocks can lead to biased results, where models applied to small caps have more observations to train on, and better conditions for learning patterns and relationships within the data. Choosing to scrape data from Yahoo Finance reduces data observations, as Yahoo only provides historical data on stocks currently listed. Stocks having been delisted either through acquisition, merger, bankruptcy or through not fulfilling requirements of the Oslo Stock Exchange before 2021 are therefore not included in the data sets. Providing the models with these additional stocks could potentially make the data sets more balanced and less biased towards small caps.     

\indent\newline
Another issue with the findings is the possibility of predictive- and portfolio performance being a result of chance and luck. Predicting the probability of a stock outperforming the cross-sectional median return of all stocks each day, represents a binary classification problem. Evaluating each stock individually can therefore be characterized as a 50/50 chance of the stock outperforming the cross-sectional median return. Running through the models thousands of times could therefore lead to a situation based on luck, where one of the RNNs have chosen stocks with abnormal large returns for a majority of the trading days. It is not unusual for a small cap share price to increase by 10-50\% in one day in situations where there have been positive news. Further, the thesis does not account for liquidity being a potential limiting factor for changes in portfolio composition, nor the portfolio size in terms of capital. There can be significant differences between managing a portfolio with 100,000 NOK and 10 million NOK in equity, especially when trading small cap stocks. A number of these will often have low liquidity, which can make it difficult to get in and out of positions. Thus, there might be several situations during backtesting where a stock is selected for trading, while in reality it would not have been possible due to low trading volume or due to the bid-ask spread being too high. When it comes to changing short-positions on a daily basis, it is important to discuss if this is a realistic trading strategy. Since the Oslo Stock Exchange is less liquid compared to other well-known stock exchanges, there are a number of stocks where short-selling is not offered as a service from brokers. This means that for the long-short portfolio and the short-only portfolio, there might be several trades where short-selling would not have been possible in reality.      

\indent\newline
In light of the research questions addressed in this thesis, the resulting evidence suggest that RNNs, and ecpecially LSTM-networks, are well-suited for predicting small cap stock returns and that they do outperform models employed to large cap stock returns, in terms of generating excess returns. However, additional data on large cap stocks needs to be included in the training and testing process to be able to validate the findings. When addressing the question of the feasibility of implementing small cap predictive models into real-life trading strategies, the answer is more complicated. Several of the issues described above needs further investigation before being able to conclude the models will have the ability to be successfully implemented in a realistic trading environment. Taking into consideration some of the limitations with the presented evidence, the small cap RNNs can be used as a supplementary tool (in addition to fundamental and/or technical analysis) for investors who seek to capitalize on the high volatility of small cap stocks.   

\indent\newline
Comparing the findings presented in this research with the efficient market hypothesis, suggest that the weak-form of the hypothesis does not hold. If small cap share prices reflect previous historical prices and returns, the recurrent neural network-models would not be able to generate excess returns and Sharpe ratios, compared to large cap stocks and relevant diversified benchmark portfolios. Reviewing the relevant performance metrics for the large cap models, suggest that the weak-form of the efficient market hypothesis does hold. This can be a result of large cap stocks being more popular among investors, where several strategies have been employed for achieving excess returns and been arbitraged away.     

\section{Further Research}
Given that the methods used in the research of this paper requires changing portfolio composition each day, a resulting negative effect is the large amount of transaction costs. A suggested solution for further research is to explore other deep learning networks that can potentially be more suited for predicting stock returns several time steps ahead. Having a model with the ability to predict stock returns for either a week or month ahead, can solve the issue of transaction costs having such a negative effect on returns, where trading frequency could be reduced substantially. Currently, a common way of predicting time series and stock returns multiple time steps ahead, is feeding the network with input features consisting of daily observations, which is used for predicting one time step ahead. The network implements this predicted value as an input feature observation to predict the next time step. This process is repeated until it reaches the time step representing the desired prediction interval. However, this method has thus far not been able to produce satisfying results due to inputs being based on model predictions, which greatly reduces predictive accuracy. Another suggested solution to be explored is splitting both input and output data into intervals of weeks or months. This seems difficult to carry out in practice as stock markets can be quite unpredictable, where a lot can happen in weeks and months, making it difficult for the deep learning network to detect and learn patterns within the data. It will also require using data stretching over a long time period. There may be other types of algorithms that are better suited for predicting stock returns with longer time steps and could be explored more closely. Other areas within the field of predicting stock returns worth investigating further, include developing more complex and intricate neural networks and deep learning models. Adding more layers to the LSTM-network (stacked ensemble) or using generative adversarial networks (GANs) could potentially lead to better results, where the models are able to comprehend deeper relationships within the data.  

\indent\newline 
Further research targeted more specifically towards predicting small stock returns involves implementing data on the bid-ask spread, as the spread between the highest bid and lowest ask tend to be relatively high in small cap stocks. Another issue with small cap trading is low trading volumes, which can potentially lead to implementation shortfall or difficulties with getting in and out of positions. Additionally, including lending fees for short-selling is another factor to explore. Thus, including arrival costs, implementation shortfall and lending fees in the backtesting process in future research can give an even more realistic assessment on how RNNs can be deployed to small cap trading. Another area to further explore is hyperparameter-optimization, where fine-tuning of different parameters can have a large effect on predictive performance and ultimately portfolio performance. Lastly, since the paper only has reviewed a time period where the majority of the period can be characterized as a bull market, future research can include more data, stretching over a longer time period in order to test network performance and portfolio performance during additional periods of bear markets as well. 