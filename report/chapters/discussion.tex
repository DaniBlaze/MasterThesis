The aim of the research in this thesis covers a complex area within the field of machine learning and deep learning.    

\section{Reliability and Validity}
Exploring the applicability of deploying recurrent neural networks in predicting small cap stock returns is fundamentally a challenging task, as future development of individual stock returns and the stock market as a whole, is difficult to predict. There are numerous factors that influence the stock market, whether it be fundamental, technical or macroeconomic factors. While the thesis has tried to accommodate several of these complex issues, it is important to discuss the reliability and validity of the resulting findings. Evidence from analyzing portfolio performance suggests that LSTMs and GRUs can be deployed to predicting small cap stock returns and achieve excess returns, compared to large cap stocks, the small cap index OSESX and the benchmark index OSEBX. However, there exists some uncertainties with the validity and reliability with these findings, where areas including data collection and methodology requires further research to be able to validate the results.      

\section{Transaction Costs}

\section{Liquidity}

\section{The 2020 Market Crash}

\section{The Efficient Market Hypothesis}


\section{Further Research}
Given that the methods used in the research of this paper requires changing portfolio composition each day, a resulting negative effect is the large amount of transaction costs. A suggested solution for further research is to explore other deep learning networks that can potentially be more suited for predicting stock returns several time steps ahead. Having a model with the ability of predicting stock returns for either a week or month ahead, can solve the issue of transaction costs having such a negative effect on returns, where trading frequency could be reduced substantially. Currently, a common way of predicting time series and stock returns multiple time steps ahead, is feedig the network with input features consisting of daily observations, which is used for predicting one time step ahead. The network implements this predicted value as an input feature observation to predict the next time step. This process is repeated until it reaches the time step representing the desired prediction interval. However, this method has thus far not been able to produce satisfying results due to inputs being based on model predictions, which greatly reduces predictive accuracy. Another suggested solution to be explored is splitting both input and output data into intervals of weeks or months. This seems difficult to carry out in practice as stock markets can be quite unpredictable, where a lot can happen in weeks and months, making it difficult for the deep learning network to detect and learn patterns within the data. There may be other types of algorithms that are better suited for predicting stock returns with longer time steps and could be explored more closely. Other areas within the field of predicting stock returns worth investigating further, include developing more complex and intricate neural networks and deep learning models. Adding more layers to the LSTM-network (stacked ensemble) or using generative adversarial networks (GANs) could potentially lead to better results, where the models are able to comprehend deeper relationships within the data.  

\indent\newline 
Further research targeted more specifically towards predicting small stock returns involves implementing data on the bid-ask spread, as the spread between the highest bid and lowest ask tend to be relatively high in small cap stocks. Another issue with small cap trading is low trading volumes, which can potentially lead to implementation shortfall or difficulties with getting in and out of positions. Additionally, including lending fees for short-selling is another factor to explore. Thus, including arrival costs, implementation shortfall and lending fees in the backtesting process in future research can give an even more realistic assessment on how RNNs can be deployed to small cap trading. Lastly, since the paper only has reviewed a period characterized as a bull market,  future research can include more data, stretching over a longer time period, in order to test network performance and portfolio performance during bear markets as well.   hyperparameter tuning