This paper explores the applicability of long short-term memory (LSTM) and gated recurrent unit (GRU) networks for predicting small cap stock returns on the Oslo Stock Exchange. The recurrent neural networks are applied to a group of small cap portfolios  and a group of large cap portfolios to assess if small cap stocks outperform large cap stocks in terms of generating excess returns. The models are trained and tested on stocks listed on the Oslo Stock Exchange from 2012-2020, where the networks predict daily out-of sample directional movements for the stocks in each group. The main independent variable consist of a sequence of previous historical returns, which implicitly tests if the weak-form of the efficient market hypothesis holds or not. 

\indent\newline
The recurrent neural networks are applied to a main trading strategy which consist of a long-only portfolio with five stocks, re-balanced on a daily basis. Two additional strategies are implemented and consist of a 50/50 long-short portfolio with two short positions and two long positions, and a short-only portfolio with four stocks. These portfolios are also re-balanced on a daily basis. The paper follows a standard procedure within the field of stock market predictions, where portfolio performance is evaluated prior to and after including transaction costs in the form of explicit costs. Six recurrent neural network variations are developed for predicting the probability of a stock outperforming the cross-sectional median return the next day. Four variations are LSTM-networks with different number of network layers and input features, while two are GRU-networks which differ in number of input features. 

\indent\newline
The resulting findings show that the small cap models significantly outperform the large cap models, in terms of predictive performance and by realizing higher annualized returns, Sharpe ratios and cumulative returns for all portfolio strategies. The GRU-variations are ranked as the best models when assessing predictive performance, where gru1\_small has an accuracy score of 58.4\%, while the large cap models have accuracy scores just above 50\%. Prior to transaction costs, the top-ranked small cap model outperforms the top ranked large cap model with Sharpe ratios of 3.17 (long portfolio), 2.67 (long-short portfolio) and 0.99 (short portfolio) with corresponding Sharpe ratios for the large cap models of 2.33, 2.17 and 1.16. After taking into account implicit transaction costs, the small cap models outperform the large cap models with Sharpe ratios of 1.92 (long portfolio), 2.17 (long-short portfolio) and -0.16 (short portfolio), compared to large cap Sharpe ratios of -1.62, -0.93 and -2.45. The small cap- and large cap benchmarks had Sharpe ratios of 0.64 and 0.76 during the same period.

\indent\newline
The results further emphasize that recurrent neural networks are well-suited for predicting small cap stock returns, where they demonstrate the ability to learn patterns within stocks with higher volatility, enabling them to capitalize on frequent trading strategies, as apposed to large cap stocks which have lower volatility and are less suited for daily trading. The presented evidence suggest the weak-form of the efficient market hypothesis does not hold for small cap stocks, where there seems to be possibilities of arbitrage. The results for the large cap models suggest that the hypothesis does hold, where previous possibilities of achieving excess returns may have been arbitraged away, due to large cap stocks being much more popular among investors, where similar strategies may have been employed in the past and present. 

\indent\newline
Despite the promising results presented in this paper, there are still areas within the research which requires further examining before being able to conclude the models can be used as a stand-alone trading strategy. For the time being, it seems the models can be applied as a supplementary trading tool for strategies that rely on technical and/or fundamental analysis.  