This paper explores the applicability of long short-term memory (LSTM) and gated recurrent unit (GRU) networks for predicting small cap stock returns on the Oslo Stock Exchange. The recurrent neural networks are applied to a group of small cap portfolios  and a group of large cap portfolios to assess if small cap stocks outperform large cap stocks in terms of generating excess returns. The models are trained and tested on stocks listed on the Oslo Stock Exchange from 2012-2020, where the networks predict daily out-of sample directional movements for the stocks in each group. The main independent variable consist of a sequence of previous historical returns, which implicitly tests if the weak-form of the efficient market hypothesis holds or not. 

\indent\newline
The recurrent neural networks are applied to a main trading strategy which consist of a long-only portfolio with five stocks, re-balanced on a daily basis. Two additional strategies are implemented and consist of a 50/50 long-short portfolio with two short positions and two long positions, and a short-only portfolio with four stocks. These portfolios are also re-balanced on a daily basis. The paper follows a standard procedure within the field of stock market predictions, where portfolio performance is evaluated prior to and after including transaction costs in the form of explicit costs. Six recurrent neural network variations are developed for predicting the probability of a stock outperforming the cross-sectional median return the next day. Four variations are LSTM-networks with different number of network layers and input features, while two are GRU-networks which differ in number of input features. 

\indent\newline
Prior to transaction costs, the resulting findings show that the small cap models significantly outperform the large cap models, in terms of predictive performance and by realizing higher annualized returns, Sharpe ratios and cumulative returns for all portfolio strategies. The GRU-variations            