Market data on all stocks listed on the Oslo Stock Exchange as of 2021 are scraped from Yahoo Finance, using the Python library $\textit{yfinance}$ and each stock's unique ticker. This main data set is scraped using a max function in yfinance, which gives the maximum available historical data on each stock. The scraped data set provides data on daily dividend-adjusted closing share prices and daily traded volume for each stock. Further, market capitalization data is downloaded from Oslo Børs, which includes each stock's market capitalization at the end of the year in the period from 2003 until 2020. This is later used for separating small cap stocks from large cap stocks, and ultimately creating two different data sets based on a market cap threshold. Additional independent variables such as the VIX, Brent oil price, 10-year US treasury rate, and the USD/NOK foreign exchange rate are scraped individually from Yahoo Finance, where they are later merged with the two data sets. 

\section{Software and Hardware}  
Data collection, data preparation and model development is conducted in Python 3.7, using Visual Studio Code as an integrated development environment (IDE). The recurrent neural networks are constructed using the library TensorFlow 1.15. This library includes the Contrib module supporting interfaces for different recurrent neural networks. Metrics are extracted using Keras, Scikit-learn and Statistics. Plots are generated using pyplot from the matplotlib library.  

\indent\newline
Creating data sets and training the neural networks require a high level of computing power. Although there exists cloud services and GPU options to increase performance, this paper conducted all executions on a personal computer. The largest data set took roughly 12 hours to construct, while training and testing the different recurrent neural networks took about 3 hours.    
