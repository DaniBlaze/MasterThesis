Artificial intelligence, and specifically machine learning, has experienced rapid growth during the last decades. Several industries and companies are increasingly implementing the power of machine learning to assist their day-to-day business operations. The stock market is no exception, where hedge funds and professional traders are making use of computational power for algorithmic trading, stock price predictions and as support for decision-making. The concept of predicting future stock price movements has been a topic of discussion for many years. One view is that future stock prices, or at least future price movements, can be predicted through technical analysis, which relies on studying past stock prices to predict future movements, or through fundamental analysis, which consists of studying companies' financial information such as earnings, equity, liabilities, cash flow etc. 

\indent \newline 
Critics argue that stock prices are unpredictable, where price movements follow a "random walk". The term is commonly used in financial literature to describe price series where changes in price represent random departures from previous prices \cite{malkiel}. The "random walk" idea is closely related to the efficient market hypothesis. The hypothesis is based on a belief that markets are completely efficient, meaning that all information and news related to individual stocks and the market as a whole are reflected in the prices without delay. An immediate stock price reflection means that future price movements can only be explained by future news flows and cannot be predicted based on historical prices nor past events. As a result of this, investors are not able to obtain a greater return than a randomly selected portfolio consisting of individual stocks and with a comparable level of risk. This also means that investors with little market experience are able to achieve the same rate of return as professional investors through holding a diversified portfolio with prices given by the market \cite{malkiel}.  

\indent \newline 
The main concepts within the efficient market hypothesis has been challenged by both technical - and fundamental analysis, where the belief is that future stock prices and price directions can be predicted. During the last years the hypothesis has been further challenged by an increasing use of artificial intelligence and machine learning as a tool for predicting stock prices. There are several different machine learning techniques that can be applied to this area of study. A common approach is called deep learning, which differs from more traditional machine learning techniques, in the form of not only being able to improve decisions based on previous learning, but also make intelligent decisions without human interference. The basis of creating a machine learning algorithm consists of splitting historical data into training sets and test sets, where the model is repeatedly trained and tested to improve accuracy and performance. The aim is for the model to be able to find underlying relationships within the data in order for the model to make predictions on new and unseen data. Artificial neural networks (ANN) is a type of deep learning which tries to resemble how the human brain functions. Stock price predicting is a complex topic, which requires intricate algorithms in order to detect underlying patterns in the data. ANNs are therefore more suitable for stock price predicting than more simple machine learning models.

\indent \newline 
There exists currently some research on stock price prediction for companies that are categorized as large cap stocks. These types of companies are normally mature companies associated with high stable earnings, and with a low level of risk with regards to the company's ability to generate future cash flow. However, investing in large cap stocks does not necessarily lead to greater returns as they often have low volatility which therefore leads to minor changes in stock price. On the other hand, companies that are categorized as small cap stocks are often associated with high volatility and large fluctuations in stock price. Research on how small cap stocks perform compared to large cap stocks dates all the way back to 1981, where Banz found evidence of a trend referred to as the "size effect" \cite{BANZ19813}. He examined common stocks listed on NYSE in the period between 1936-1975 and found that smaller firms had, on average, higher risk adjusted returns than larger firms. The evidence has been further confirmed with similar research carried out by Bauman et al. in 1998, in which they examined data from 21 international markets in the period between 1986-1996 and found the same “size effect” for markets outside of the US \cite{bauman}. In a more recent research article from 2019 Rompotis, Gerasimos explores the same topic of interest with data on 66 large-cap and 34 small-cap USA-listed ETFs (Exchange Traded Funds) during the period 2012-2016. The findings support previous evidence of the “size effect”, but a higher level of risk is highlighted as a factor related to small cap ETFs \cite{rompotis}.

\indent \newline 
The main contribution of the following paper consists of examining if it is possible to apply the same logic and techniques used for predicting stock returns for large cap companies to predict stock returns for small cap companies. The paper aims to advance current literature within the field, where the research scope is extended to include both large cap stocks and small cap stocks. The paper will incorporate several deep learning algorithms for predicting stock returns, and it will present findings on how well the models perform on both groups of stocks and see if small cap trading results in greater returns than large cap trading. Given that previous research suggests a higher return associated with holding a portfolio consisting of small cap stocks, the following research can potentially advance the field of stock predictions, reduce the added risk in small cap trading, and increase returns from stock investments.  

\section{Problem Definition}
The recent development within artificial intelligence and deep learning, as well as the increasing use within several sectors and industries, presents opportunities within the field of finance to further explore the applicability of prediction models on stock investments. The aim of the thesis consists of developing several deep learning algorithms where model performance is evaluated on both small cap- and large cap stocks. The scope of the paper is set to two groups of stocks listed on the Oslo Stock Exchange, which can be characterized as small cap- and large cap stocks. The paper will also discuss how the prediction models can be implemented in a real-life trading strategy, and explore the potential complications of small cap trading. The research will be of interest for both academics and institutions within the financial industry, and aims to answer the following research questions:  

\indent \newline
\begin{enumerate}
\item \textit{How well do neural networks predict price movements for small cap stocks on the Oslo Stock Exchange, and do they outperform large cap stocks in terms of greater returns?}
\item \textit{Is it feasible to implement small cap predictive models into a real-life trading strategy?}
\end{enumerate}

\indent \newline
The trading strategy the models will be tested on consists of long positions, which means that short strategies are not included in the scope of the paper. The reasoning behind this is that the possibility of shorting will greatly complicate the research and go beyond the limitations of this paper. 

\section{Literature Review}
Applying deep learning and artificial neural networks to financial markets and trading is a relatively new topic of interest, and the current research is mainly focused on predicting large cap stock returns. To the best of my knowledge there does not exist any literature on predicting stock returns for small cap stocks listed on the Oslo Stock Exchange, where model performance and stock returns are compared to large cap stocks. The following section presents previous research on deep learning and stock return predictions, and emphasizes how the paper seeks to advance the current literature and in which areas.     

\indent\newline
Even though the topic of interest only has experienced increased focus during the last few years, research on predicting stock returns using deep learning dates back to 1988, where Halbert White examined neural network modelling for decoding nonlinear regularities in asset price movements \cite{white}. White’s article challenged the view of markets being efficients, based on the idea that when technology evolves (e.g. development within artificial intelligence), former common human beliefs can be questioned. The article assessed daily returns from IBM common stocks, where 1000 observations were used for training purposes, and 500 before- and after samples were used to evaluate the neural network. The network can be characterized as a “simple” network, where the resulting findings were not able to find evidence against the simple efficient market hypothesis due to overfitting issues. Despite not being able to disconfirm the simple efficient market hypothesis, White’s research indicated a possibility of utilizing deep learning as a tool for exploiting potential inefficiencies in stock markets.  

\indent\newline
In 1993 Kryzanowski et al. contributed to the literature with an article on artificial neural networks, where they examined if the network was able to discriminate between stocks that provide positive returns and stocks that provide negative returns \cite{kryz}. A neural network was trained to learn the relationships between a company's stock return one year forward, financial data on the company and it's industry, and macroeconomic variables. The neural network consisted of a pattern-recognition algorithm and can be termed as a classification model. The resulting findings showed that the neural network was able to correctly classify 72\% of the positive/negative returns, meaning that the classification algorithm showed promising potential for stock picking. However, since the data consisted of a limited sample of smaller companies over a short time period, the results remained open for discussion.

\indent\newline
In more recent work, Krauss et al. advanced the literature with two articles published in 2017 and 2018. In 2017, Krauss, Do and Huck published research which tried to bridge the gap between academic finance and the financial industry, where academic finance had started to focus on transparency and lower frequency data at the expense of performance, while the financial industry had started to focus their attention on performance and high frequency data at the expense of transparency \cite{huck}. The authors deployed deep neural networks, gradient-boosted trees and random forests on all of the S\&P 500 index constituents, with an aim of predicting the probability for each stock to outperform the general market. The findings showed that the neural networks did not outperform the other machine learning models, and also that the models performed particularly well in situations of market turmoil, in other words, when the markets are influenced by financial crisis, bubbles etc. Another interesting finding was that after the year of 2001 daily returns declined. The authors reasoned this with an increased usage of machine learning algorithms during this period. 

\indent\newline
In the following year, Krauss and Fischer further examined the use of deep learning algorithms for financial time series predictions \cite{krauss}. The authors deployed long short-term memory (LSTM) for predicting out-of-sample directional movements for the constituents stocks of the S\&P 500, using data from the period between 1992-2015. Long short-term memory is a type of artificial recurrent neural network (RNN) and differs from more standard feed forward neural networks in terms of having feedback connections. This will be further explained in the theory section. The research was an extension from the research carried out the year before, where the aim was to examine if  a recurrent neural network could outperform more traditional machine learning algorithms, which consisted of memory free classification methods such as random forest, deep neural network, and logistic regression classifier. The article presented evidence of the recurrent neural network outperforming all of the other models with a sharpe ratio of 5.8, as well as having the highest model accuracy. Much like the previous research article market efficiency started to rise after 2010, which resulted in a decline in the network's profitability. The research of Krauss et al. points to a well-known theory consisting of mean-reversion, which means that asset prices and historical returns revert to the mean of the entire data set. The findings questions the ability of achieving greater returns (over a longer period of time) associated with applying prediction models on financial time series.

\indent\newline
There has previously not been extensive research on how deep learning can be applied to stock return predictions on the Oslo Stock Exchange. However, in the last few years there have been a few articles on the subject. In 2016, Olden conducted research on whether or not it is possible to use machine learning algorithms to make a profitable stock trading scheme from stocks listed on the Oslo Stock Exchange \cite{olden}. He applied a stacked ensemble learning technique, which consisted of combining several machine learning algorithms into a single algorithm, and attempted to predict daily movements of the 22 stocks with the highest turnover on the Oslo Stock Exchange, using a total of 37 machine learning techniques. The research did not find concluding evidence of achieving a profitable scheme over a longer period of time, nor that stacked ensemble techniques outperforms other machine learning techniques. However, the top performing algorithms were able to outperform the Oslo Benchmark Index during the time period.    

\indent\newline
Two years later, Lund and Løvås conducted similar research by employing deep learning for stock return prediction on the Oslo Stock Exchange \cite{lund}. Their paper was inspired by the work of Krauss and Fischer, where they wanted to add on the previous research on long short-term memory networks by testing similar methods and techniques on a less liquid stock exchange. As an additional contribution to the existing literature they examined the models' applicability to real-life trading, in terms of evaluating model performance when taking into account transaction costs, strategies related to reducing transaction costs, and the bid-ask spread. The resulting Sharpe ratio of applying the long short-term memory model on simple long trading strategies before transaction costs was 3.25, whereas the Oslo Stock Exchange Benchmark Index had a Sharpe ratio of 0.30. The long short-term memory model outperformed all of the other benchmark models when looking at model performance and accuracy. As in previous research, the model experienced declining excess returns towards the last years of the period studied, which points to mean-reversion. It does however differ from previous studies, in terms of being able to have excess returns throughout the whole period. When including transaction costs, the findings show that excess returns are lost in the bid-ask spread, and the model is only able to produce a Sharpe ratio of 0.37. 




